\documentclass[11pt, a4paper]{report}

\usepackage[utf8]{inputenc}
\usepackage[margin=0.5in]{geometry}
\usepackage{fancyvrb}
\usepackage{amsmath}
\usepackage{amsfonts}
\usepackage{amssymb}
\usepackage{enumitem}
\usepackage{listings}
\newcommand{\precprec}{\prec\mathrel{\mkern-5mu}\prec}
\begin{document}
\title{Tarea 5: \\Lenguajes de Programaci\'on}
\author{
  Araujo Chávez Mauricio\\
  \texttt{312210047}
  \and
  Carmona Mendoza Mart\'in\\
  \texttt{313075977}
}
\date{}
\maketitle

\section*{Considera la gramática descrita en la tarea pasada.}
\begin{enumerate}
	\item 
		\begin{itemize}
		 	\item Extiende la gramatica de la tarea pasada con los marcos de operación de las expresiones error y try e1 catch e2 vistas en clase y describe los marcos de operacion (no es necesario describir los marcos descritos en la tarea pasada).
		 	\\\textbf{error}
		 	\\Denotamos error como un nuevo estado de la máquina denotado por
		 	$$ \mathcal{P} \precprec error$$
		 	Donde el erorr va vaciando la pila hasta encontrar un manejador y dado el caso que no existe la expresión anterior se considera como estado final.
		 	$$ Donde\ \mathcal{P}\ es\ la\ pila.$$
		 	\textbf{try e1 catch e2}
		 	\\Se ejecuta e1 y se espera un valor para manejar el error si es que existe.
		 	$$ \frac{}{catch(-,e_{2})\ marco} $$
		 	\item Escribe cuatro programas utilizando la gramatica descrita. Todos deben estar contenidos en un try catch, dos de ellos deben arrojar un error (y manejarlo) y dos de ellos deben terminar sin utilizar el marco del catch. Entre los cuatro programas debes utilizar todos las expresiones del lenguaje.
		 		\begin{itemize}
		 			\item $p_{1} \leftrightharpoons$
		 			\begin{lstlisting}
	try
		let x = 2 in
			x + (if not(x < 3) then 10 else error)
		end
	catch error
		show "El numero es menor que 3"
		 			\end{lstlisting}
		 			\item $p_{2} \leftrightharpoons$
		 			\begin{lstlisting}[mathescape=true]
	try
		app(if(($\lambda$x.(not x)true) 
			then $\lambda$y.(y+6) 
			else $\lambda$z(z+0)
			,7)
	catch error
		show "Valor incorrecto"
		 			\end{lstlisting}
		 		\end{itemize}
		 	\item Realiza la ejecucion de los cuatro programas, puedes obviar algunas operaciones excepto aquellas que involucran la expresion try-catch.
		 		\begin{itemize}
		 			\item $p_{1} \leftrightharpoons$
		 				\\Definimos e1 como let x = 2 in x + (if not(x $<$ 3) then 10 else error) end\\
		 				y e2 como show "El numero es menor que 3"
		 				$$ \Box \succ p_{1} \rightarrow_{\mathcal{K}}$$
		 				$$ catch(-,e_{2});\Box \succ e_{1} \rightarrow_{\mathcal{K}}$$
		 				Definimos l como x + (if not(x $<$ 3) then 10 else 30)
		 				$$ (-,x.l),catch(-,e_{2});\Box \succ 2 \rightarrow_{\mathcal{K}} $$
		 				$$ (-,x.l),catch(-,e_{2});\Box \prec 2 \rightarrow_{\mathcal{K}} $$
		 				Obtenemos 2 + (if not(2 $<$ 3) then 10 else error)
		 				$$ (2,-),catch(-,e_{2});\Box \prec (if\ not(2 < 3)\ then\ 10\ else\ error) \rightarrow_{\mathcal{K}}$$
		 				$$ if(-,10,error);(2,-),catch(-,e_{2});\Box \prec not(2<3) \rightarrow_{\mathcal{K}}$$
		 				Al tener valores omitimos pasos triviales de $<$
		 				$$ if(-,10,error);(2,-),catch(-,e_{2});\Box \prec not\ true \rightarrow_{\mathcal{K}}$$
		 				$$ if(-,10,error);(2,-),catch(-,e_{2});\Box \succ false \rightarrow_{\mathcal{K}}$$
		 				$$ if(-,10,error);(2,-),catch(-,e_{2});\Box \prec false \rightarrow_{\mathcal{K}} $$
		 				$$ (2,-),catch(-,e_{2});\Box \succ error \rightarrow_{\mathcal{K}}$$
		 				Propagamos el error
		 				$$ (2,-),catch(-,e_{2});\Box \precprec error \rightarrow_{\mathcal{K}} $$
		 				Y al tener $catch(-,e_{2})$ podemos continuar con la ejecución de $e_{2}$
		 				$$ catch(error,-)\Box \succ e_{2} \rightarrow_{\mathcal{K}} $$
		 				Intuitivamente, el programa imprimirá la leyenda.
		 				$$ \Box \succ "El\ numero\ es\ menor\ que\ 3" $$
		 			\item $p_{2} \leftrightharpoons$
		 				\\Definimos e1 como app(if($\lambda$x.(not x)true) then $\lambda$y.(y+x) else $\lambda$z(z+),7)\\
		 				y e2 como show "Valor incorrecto"
		 				$$ \Box \succ p_{2} \rightarrow_{\mathcal{K}}$$
		 				$$ catch(-,e_{2}); \Box \succ e_{1} \rightarrow_{\mathcal{K}}$$
		 				Definimos l como if($\lambda$x.(not x)true) then $\lambda$y.(y+x) else $\lambda$z(z+0)
		 				$$ app(-,7),catch(-,e_{2}); \Box \succ l \rightarrow_{\mathcal{K}} $$
		 				$$ if(-,\lambda y.(y+x),\lambda z(z+0);app(-,7),catch(-,e_{2}); \Box \succ (\lambda x.(not\ x))true \rightarrow_{\mathcal{K}}  $$
		 				$$ app(-,true);if(-,\lambda y.(y+x),\lambda z(z+0);app(-,7),catch(-,e_{2}); \Box \succ \lambda x.(not\ x) \rightarrow_{\mathcal{K}}$$
		 				$$ app(-,true);if(-,\lambda y.(y+x),\lambda z(z+0);app(-,7),catch(-,e_{2}); \Box \prec \lambda x.(not\ x) \rightarrow_{\mathcal{K}} $$
		 				$$ app(\lambda x.(not\ x),-);if(-,\lambda y.(y+x),\lambda z(z+0);app(-,7),catch(-,e_{2}); \Box \succ true  \rightarrow_{\mathcal{K}} $$
		 				$$ app(\lambda x.(not\ x),-);if(-,\lambda y.(y+x),\lambda z(z+0);app(-,7),catch(-,e_{2}); \Box \prec true  \rightarrow_{\mathcal{K}} $$
		 				$$ if(-,\lambda y.(y+x),\lambda z(z+0);app(-,7),catch(-,e_{2}); \Box \succ (not\ x)[x\ :=\ true]  \rightarrow_{\mathcal{K}} $$
		 				$$ if(-,\lambda y.(y+x),\lambda z(z+0);app(-,7),catch(-,e_{2}); \Box \succ not\ true \rightarrow_{\mathcal{K}} $$
		 				$$ not(-);if(-,\lambda y.(y+x),\lambda z(z+0);app(-,7),catch(-,e_{2}); \Box \succ true \rightarrow_{\mathcal{K}} $$
		 				$$ not(-);if(-,\lambda y.(y+x),\lambda z(z+0);app(-,7),catch(-,e_{2}); \Box \prec true \rightarrow_{\mathcal{K}} $$		
		 				$$ if(-,\lambda y.(y+x),\lambda z(z+0);app(-,7),catch(-,e_{2}); \Box \prec false \rightarrow_{\mathcal{K}} $$
		 				$$ app(-,7),catch(-,e_{2}); \Box \succ \lambda z(z+0) \rightarrow_{\mathcal{K}} $$
		 				$$ app(-,7),catch(-,e_{2}); \Box \prec \lambda z(z+0) \rightarrow_{\mathcal{K}} $$
		 				$$ app(\lambda z(z+0),),catch(-,e_{2}); \Box \succ 7 \rightarrow_{\mathcal{K}} $$
		 				$$ app(\lambda z(z+0),),catch(-,e_{2}); \Box \prec 7 \rightarrow_{\mathcal{K}} $$
		 				$$ catch(-,e_{2});\Box \succ (z\ + \ 0)[z\ :=\ 7] $$
		 				Trivialmente sumamos 7 + 0
		 				$$ catch(-,e_{2});\Box \succ 7 \rightarrow_{\mathcal{K}} $$
		 				Al devolver un valor e1 se elimina el marco manejador y sigue ejecución sin salto.
		 				$$ \Box \succ 7  $$ 
		 		\end{itemize}
		 \end{itemize} 
	\item
		\begin{itemize}
			\item Extiende la gramatica de la tarea pasada con los marcos de operación de las expresiones $raise(e)$ y $handle\ e1\ with\ x\Rightarrow e2$ vistas en clase y describe los marcos de operacion (no es necesario describir los marcos descritos en la tarea pasada).
			\\\textbf{raise(e)}
			\\Lanzamos el error esperando un valor de e.
			$$ \frac{}{raise(-)\ marco} $$
			\textbf{handle e1 with x $\Rightarrow$ e2}
			\\Manejamos un error con x aplicandolo a e2.
			$$ \frac{}{handle(-,x.e_{2})\ marco} $$
			\item Extiende el comportamiento de la maquina $\mathcal{K}$
			\\Extendemos de la siguiente manera.
			\\\textbf{raise}
			$$ \frac{}{\mathcal{P} \succ raise(e) \rightarrow_{\mathcal{K}} raise(-);\ \mathcal{P} \succ e}$$			
			$$ \frac{}{raise(-);\ \mathcal{P} \prec v \rightarrow_{\mathcal{K}} \mathcal{P} \precprec raise(v)}$$
			\textbf{handle}
			$$ \frac{}{\mathcal{P} \succ handle(e_{1},x.e_{2}) \rightarrow_{\mathcal{K}} handle(-,x.e_{2});\ \mathcal{P} \succ e_{1} } $$
			$$ \frac{}{handle(-,x.e_{2});\ \mathcal{P} \prec v \rightarrow_{\mathcal{K}} \mathcal{P} \prec v } $$
			$$ \frac{m \neq handle(-,x.e_{2})}{m;\mathcal{P} \precprec raise(v) \rightarrow_{\mathcal{K}} \mathcal{P} \precprec raise(v) } $$
			$$ \frac{}{handle(-,x.e_{2});\mathcal{P} \precprec raise(v) \rightarrow_{\mathcal{K}} \mathcal{P} \succ e_{2}[x\ :=\ v] } $$
			\item Escribe cuatro programas utilizando la gramatica descrita. Todos deben estar contenidos en un handle, dos de ellos deben arrojar un error con valor (y manejarlo) y dos de ellos deben terminar sin utilizar el manejador. Entre los cuatro programas debes utilizar todos las expresiones del lenguaje.
			\item Realiza la ejecucion de los cuatro programas, puedes obviar algunas operaciones excepto aquellas que involucran la expresiones raise y handle.
		\end{itemize}
\end{enumerate}
\end{document}