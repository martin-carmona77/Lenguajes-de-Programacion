\documentclass[11pt, a4paper]{report}

\usepackage[utf8]{inputenc}
\usepackage[margin=0.5in]{geometry}
\usepackage{fancyvrb}
\usepackage{amsmath}
\usepackage{amsfonts}
\usepackage{amssymb}
\usepackage{enumitem}

\begin{document}
\title{Tarea 1: \\Lenguajes de Programaci\'on}
\author{
  Araujo Chavez Mauricio\\
  \texttt{312210047}
  \and
  Carmona Mendoza Mat\'in\\
  \texttt{313075977}
}
\date{}
\maketitle
\section*{1. Los naturales de Church se definen como sigue:}
 \begin{align*}
   0 &= \lambda s.\lambda z.z\\
	1 &= \lambda s.\lambda z.s\;z\\
	2 &= \lambda s.\lambda z.s (s\;z)\\
        3 &= \lambda s.\lambda z.s(s (s\;z))\\
	\vdots 
 \end{align*}
 Se define el par ordenado como \underline{pair}:$= \lambda x.\lambda y. \lambda p. p\;x\;y$
 , así el par ordenado (a,b)$= \underline{pair}\;\; a\;b = \lambda p. p\;a\;b$ Las funciones 
 para obtener la primer y segunda componente de un par ordenado se definen respetivamente 
 como: \underline{fst}:$= \lambda p. p\;\underline{true}$ y \underline{snd}:$= \lambda p. 
 p\;\underline{false}$\\ 
 Sean $g_{1}$, $h_{1}$ las siguientes funciones:
  \begin{center}
	$g_{1}:= \lambda n. \lambda s. \lambda z. n\;(\lambda h_{1}. \lambda h_{2}.\;h2\;
    (h1\;s)) \;(\lambda u.z)\;(\lambda u.u)$\\
	$h_{1}:=\lambda n.\underline{fst}\;(n\;\underline{ss}\;\underline{zz})$, donde 
    \underline{ss} = $\lambda p.\underline{pair}\;(\underline{snd}\;p)\;
    (\underline{suc}\;(\underline{snd}\;p))$, y $\underline{zz} := \underline{pair}\;0\;0$\\
  \end{center}
\begin{enumerate}[label=\alph*)]

\item Calcula $(g_{1}\; 0)$ y $(g_{1}\; 3)$
	\begin{itemize}
	\item $(g_{1}\; 0)$ \\
    $\lambda n.\lambda s. \lambda z. n (\lambda h1. \lambda h2. h2\;(h1\;s)) (\lambda 
    u.z) (\lambda u. u)\; 0 =$\\
    $\lambda s. \lambda z. 0 (\lambda h1. \lambda h2. h2\;(h1\;s)) (\lambda 
    u.z) (\lambda u. u)\; =$\\
    $\lambda s. \lambda z. (\lambda s.\lambda z. z) (\lambda h1. \lambda h2. h2\;(h1\;s)) 
    (\lambda u.z) (\lambda u. u)\; =$\\
    $\lambda s. \lambda z. (\lambda z. z) (\lambda u.z) (\lambda u. u)\; =$\\
    $\lambda s. \lambda z. (\lambda u.z) (\lambda u. u)\; =$\\
    $\lambda s. \lambda z. z\; = 0$\\
    
    \item $(g_{1}\; 3)$ \\
    $\lambda n.\lambda s. \lambda z. n (\lambda h1. \lambda h2. h2\;(h1\;s)) (\lambda 
    u.z) (\lambda u. u)\; 3 =$\\
    $\lambda s. \lambda z. 3 (\lambda h1. \lambda h2. h2\;(h1\;s)) (\lambda 
    u.z) (\lambda u. u)=$\\
    $\lambda s. \lambda z. (\lambda s.\lambda z. s(s(s\;z))) 
    (\lambda h1. \lambda h2. h2\;(h1\;s)) (\lambda u.z) (\lambda u. u)=$\\
    $\lambda s. \lambda z. (\lambda s^{2}.\lambda z^{2}. s^{2}(s^{2}(s^{2}\;z^{2}))) 
    (\lambda h1. \lambda h2. h2\;(h1\;s)) (\lambda u.z) (\lambda u^{2}. u^{2})=$\\
    $\lambda s. \lambda z. (\lambda z^{2}. ((\lambda h1. \lambda h2. h2\;(h1\;s)))
    (((\lambda h1. \lambda h2. h2\;(h1\;s)))(((\lambda h1. \lambda h2. h2\;
    (h1\;s)))\;z^{2}))) (\lambda u.z) (\lambda u^{2}. u^{2})=$\\
    $\lambda s. \lambda z. (\lambda z^{2}. ((\lambda h1. \lambda h2. h2\;(h1\;s)))
    (((\lambda h1^{2}. \lambda h2^{2}. h2^{2}\;(h1^{2}\;s)))\\
    (((\lambda h1^{3}. \lambda h2^{3}. h2^{3}\; (h1^{3}\;s)))\;z^{2}))) (\lambda u.z) 
    (\lambda u^{2}. u^{2})=$\\
    $\lambda s. \lambda z. ((\lambda h1. \lambda h2. h2\;(h1\;s)))
    (((\lambda h1^{2}. \lambda h2^{2}. h2^{2}\;(h1^{2}\;s)))(((\lambda h1^{3}. \lambda 
    h2^{3}. h2^{3}\; (h1^{3}\;s)))\; (\lambda u.z))) (\lambda u^{2}. u^{2})=$\\
    $\lambda s. \lambda z. ((\lambda h2. h2\;((((\lambda h1^{2}. \lambda h2^{2}. h2^{2}\;
    (h1^{2}\;s)))(((\lambda h1^{3}. \lambda h2^{3}. h2^{3}\; (h1^{3}\;s)))\; 
    (\lambda u.z)))s))) (\lambda u^{2}. u^{2})=$\\
    $\lambda s. \lambda z. (((\lambda u^{2}. u^{2})\;((((\lambda h1^{2}. \lambda h2^{2}. 
    h2^{2}\; (h1^{2}\;s)))(((\lambda h1^{3}. \lambda h2^{3}. h2^{3}\; (h1^{3}\;s)))\; 
    (\lambda u.z)))s)))=$\\
    $\lambda s. \lambda z. ((\lambda h1^{2}. \lambda h2^{2}. 
    h2^{2}\; (h1^{2}\;s)))(((\lambda h1^{3}. \lambda h2^{3}. h2^{3}\; (h1^{3}\;s)))\; 
    (\lambda u.z)))s)))=$\\
    $\lambda s. \lambda z. ((\lambda h2^{2}. 
    h2^{2}\; ((((\lambda h1^{3}. \lambda h2^{3}. h2^{3}\; (h1^{3}\;s)))\; 
    (\lambda u.z)))s)))\;s)))=$\\
    $\lambda s. \lambda z. ((
    s\; ((((\lambda h1^{3}. \lambda h2^{3}. h2^{3}\; (h1^{3}\;s)))\; 
    (\lambda u.z)))s))=$\\
    $\lambda s. \lambda z.(s ((((\lambda h2^{3}. h2^{3}\; ((\lambda u.z)\;s)))s))=$\\
    $\lambda s. \lambda z.(s (s((\lambda u.z)\;s)))=$\\
    $\lambda s. \lambda z.(s (s\; z))= 2$\\
	\end{itemize}
\item Calcula $(h_{1}\; 1)$ y $(h_{1}\; 2)$
	\begin{itemize}
	\item $(h_{1}\; 1)$ \\
    $(\lambda n. \underline{fst} (n\;\underline{ss}\;\underline{zz}))\;1=$\\
    $(\underline{fst} (1\;\underline{ss}\;\underline{zz}))=$\\
    $(\underline{fst} ((\lambda s.\lambda z. s\;z)\;\underline{ss}\;\underline{zz}))=$\\
    $(\underline{fst} ((\lambda z. \underline{ss}\;z)\;\underline{zz}))=$\\
    $\underline{fst} (\underline{ss}\;\underline{zz})=$\\
    $\underline{fst} (\underline{ss}\;(\underline{pair}\;0\;0))=$\\
    $\underline{fst} (\underline{ss}\;(\lambda p.p\;0\;0))=$\\
    $\underline{fst} (\lambda p.\underline{pair} (\underline{snd}\;p) (\underline{suc} 
    (\underline{snd}\;p))\;(\lambda p.p\;0\;0))=$\\
    $\underline{fst} (\underline{pair} (\underline{snd}\;
    (\lambda p.p\;0\;0)) (\underline{suc}(\underline{snd}\;(\lambda p.p\;0\;0))))=$\\
    $\underline{fst} (\underline{pair}\;0\; 
    (\underline{suc}(\underline{snd}\;(\lambda p.p\;0\;0))))=$\\
    $\underline{fst} (\underline{pair}\;0\; 
    (\underline{suc}\;0))=$\\
    $\underline{fst} (\underline{pair}\;0\;1)=$\\
    $\underline{fst} (\lambda p. p\;0\;1)= 0$\\
    
   	\item $(h_{1}\; 2)$ \\
    $(\lambda n. \underline{fst} (n\;\underline{ss}\;\underline{zz}))\;2=$\\
    $\underline{fst} (2\;\underline{ss}\;\underline{zz}))=$\\
    $\underline{fst} (((\lambda s.\lambda z. 
    s(s\;z))\;\underline{ss}\;\underline{zz})=$\\
    $\underline{fst} ((\lambda z. \underline{ss}(\underline{ss}\;z))
    \;\underline{zz}))=$\\
    $\underline{fst} (\underline{ss}(\underline{ss}\;\underline{zz}))=$\\
    $\underline{fst} (\underline{ss}(\underline{pair}\;0\;1))=$\\
    $\underline{fst} ((\lambda p.\underline{pair} (\underline{snd}\;p) (\underline{suc} 
    (\underline{snd}\;p))(\underline{pair}\;0\;1))=$\\
    $\underline{fst} (\underline{pair} (\underline{snd}\;				
    (\underline{pair}\;0\;1)) (\underline{suc} 
    (\underline{snd}\;(\underline{pair}\;0\;1))))=$\\
    $\underline{fst} (\underline{pair} (\underline{snd}\;				
    (\underline{pair}\;0\;1)) (\underline{suc}\;1))=$\\
    $\underline{fst} (\underline{pair} (\underline{snd}\;				
    (\underline{pair}\;0\;1))\;2)=$\\
    $\underline{fst} (\underline{pair}\;1\;2)= 1$\\
    
    %donde ss = λp.pair (snd p) (suc (snd p)), y zz := pair 0 0
    %ss%
	\end{itemize}
\item ¿Qué hacen las funciones $g_{1}$ y $h_{1}$?
	
  Las funciones $g_{1}$ y $h_{1}$ calculan el antecesor de un número (En caso de que el n\'umero sea 0
  la funci\'on regresa 0).

\end{enumerate}

\section*{2. Los naturales de Scott se definen como sigue:}
 \begin{align*}
	0 &= \lambda x.\lambda y. x \\
	1 &= \lambda x.\lambda y. y\;0 \\
	2 &= \lambda x.\lambda y. y\;1 \\
    3 &= \lambda x.\lambda y. y\;2 \\
	\vdots 
  \end{align*}
  Sean $f_{2}$, $g_{2}$ y $h_{2}$ las siguientes funciones:
   \begin{center}
	$f_{2}:= \lambda n. \lambda x. \lambda y. y\; n$\\
	$g_{2}:= \lambda n. n\; 0\; (\lambda x. x)$\\
	$h_{2}:= \lambda n. n \; \underline{true} \; (\lambda x. \underline{false})$\\
  \end{center}
  
  \begin{enumerate}[label=\alph*)]
	\item Calcula $(f_{2}\;0)$ y $(f_{2}\;3)$
    \begin{itemize}
    \item $(f_{2}\;0)$ \\
    $(\lambda n. \lambda x. \lambda y. y\; n)\;0=$\\
    $\lambda x. \lambda y. y\;0=$\\
    $\lambda x. \lambda y. y\;(\lambda x.\lambda y. x)=1$\\
    
    \item $(f_{2}\;3)$ \\
    $(\lambda n. \lambda x. \lambda y. y\; n)\;3=$\\
    $\lambda x. \lambda y. y (\lambda x. \lambda y. y\; 2)=$\\
    $\lambda x. \lambda y. y (\lambda x. \lambda y. y\; (\lambda x. \lambda y. y\; 1))=$\\
    $\lambda x. \lambda y. y (\lambda x. \lambda y. y\; 
    (\lambda x. \lambda y. y\; (\lambda x. \lambda y. y\; 0)))=$\\
    $\lambda x. \lambda y. y (\lambda x. \lambda y. y\; 
    (\lambda x. \lambda y. y\; (\lambda x. \lambda y. y\; (\lambda x. \lambda y.x))))= 4$\\
    
    \end{itemize}
	\item Calcula $(g_{2}\; 1)$ y $(g_{2}\; 4)$
    \begin{itemize}
    \item $(g_{2}\; 1)$\\
    $(\lambda n. n\; 0\; (\lambda x. x))\;1$\\
    $(1\;0\;(\lambda x. x))=$\\
    $((\lambda x. \lambda y. y\; (\lambda x. \lambda y.x))\;0\;(\lambda x. x))=$\\
    $((\lambda y. y\; (\lambda x. \lambda y.x))\;(\lambda x. x))=$\\
    $(\lambda x. x)\; (\lambda x. \lambda y.x)=$\\
    $(\lambda x. \lambda y.x)= 0$\
    
    \item $(g_{2}\; 4)$\\
    $(\lambda n. n\; 0\; (\lambda x. x))\;4$\\
    $(4\; 0\; (\lambda x. x))\;1$\\
    $((\lambda x. \lambda y. y\; (\lambda x. \lambda y. y\; (\lambda x. \lambda y. y 
    (\lambda x. \lambda y. y (\lambda x. \lambda y. x)))))
    \;0\;(\lambda x. x))=$\\
    $((\lambda y. y\; (\lambda x. \lambda y. y\; (\lambda x. \lambda y. y 
    (\lambda x. \lambda y. y (\lambda x. \lambda y. x)))))
    \;(\lambda x. x))=$\\
    $(\lambda x. x)\;(\lambda x. \lambda y. y\; (\lambda x. \lambda y. y 
    (\lambda x. \lambda y. y (\lambda x. \lambda y. x))))=$\\
    $(\lambda x. \lambda y. y\; (\lambda x. \lambda y. y 
    (\lambda x. \lambda y. y (\lambda x. \lambda y. x))))= 3$\\
    
    
    \end{itemize}
	\item Calcula $(h_{2} \;0)$ y $(h_{2}\;5)$
    \begin{itemize}
    \item $(h_{2} \;0)$ \\
    $(\lambda n. n \; \underline{true} \; (\lambda x. \underline{false}))\;0 =$\\
    $(0 \; \underline{true} \; (\lambda x. \underline{false})) =$\\
    $((\lambda x. \lambda y. x) \; \underline{true} \; (\lambda x. \underline{false})) =$\\
    $((\lambda y. \underline{true}) \; (\lambda x. \underline{false})) = 
    \underline{true}$\\
    
    \item $(h_{2} \;5)$ \\
    $(\lambda n. n \; \underline{true} \; (\lambda x. \underline{false}))\;5 =$\\
    $(5\; \underline{true} \; (\lambda x. \underline{false}))=$\\
    $((\lambda x.\lambda y. y (\lambda x.\lambda y. y (\lambda x. \lambda y. y 
    (\lambda x.\lambda y. y (\lambda x. \lambda y. y (\lambda x. \lambda y. x))))))\; 
    \underline{true} \; (\lambda x. \underline{false}))=$\\
    $((\lambda y. y (\lambda x.\lambda y. y (\lambda x. \lambda y. y 
    (\lambda x.\lambda y. y (\lambda x. \lambda y. y (\lambda x. \lambda y. x))))))
    \; (\lambda x. \underline{false}))=$\\
    $(((\lambda x. \underline{false})(\lambda x.\lambda y. y (\lambda x. \lambda y. y 
    (\lambda x.\lambda y. y (\lambda x. \lambda y. y (\lambda x. \lambda y. x)))))))= 	
    \underline{false}$\\
    
    \end{itemize}
	\item ¿Qué hacen las funciones $f_{2}$, $g_{2}$ y $h_{2}$?
    \begin{itemize}
    \item La función $f_{2}$ nos da el sucesor del número $n$
    \item La función $g_{2}$ nos da el predecesor del número $n$
    \item La función $h_{2}$ regresa \underline{true} si $n$ es cero y \underline{false} si 
    $n$ no es cero.
    \end{itemize}
    \item (Extra [+1 punto]) Haz una función que haga la suma de naturales de Scott\\\\
    $ V\ (Combinador\ Turing): (\lambda x.\lambda y.(y((xx)y)))\ (\lambda x.\lambda y.
    (y((xx)y))) $\\
    $ suc\ (sucesor): \lambda n.\lambda x.\lambda y.y\ n $\\
    $ sumaScott: V\ (\lambda f.\lambda m.\lambda n.(m\ n)\ (\lambda m'.\text{\underline{suc}}\ ((f\ m')\ n))) $
  \end{enumerate}
  
  \section*{4. Utilizando un combinador de punto fijo, implementa estas funciones de forma recursiva:}
  \begin{enumerate}[label=\alph*)]
    \item \textbf{Una función que dados n y m calcule $ n^{m} $.}\\\\
      Sobre los naturales de Church
      
      Para esto utilizamos las siguientes funciones, definidas de la siguiente manera:; \\\\
    $ ift\ (if-then-else): \lambda b.\lambda t.\lambda e.(b\ t)\ e $\\\\
    $ true: \lambda x.\lambda y.x $ \\\\
    $ false: \lambda x.\lambda y.y $ \\\\
    $ esCero?: \lambda n.(n\ (\lambda x.\text{\underline{false}}))\ \text{\underline{true}}$\\\\
    $ suc\ (sucesor): \lambda n.\lambda s.\lambda z.s\ ((n\ s)\ z) $\\\\
    $ suma: \lambda n.\lambda m.(n\ \text{\underline{suc}})\ m $\\\\
    $ prod\ (producto): \lambda n.\lambda m.(n\ (\text{\underline{suma}}\ m))\ (\lambda s.\lambda z.z) $\\\\
    $ Y\ (Combinador\ Y): \lambda f.(\lambda x.f(x\ x))\ (\lambda x.f(x\ x)) $\\\\
    $ pred\ (predecesor): g_{1}\ \text{(1er Ejercicio)}$\\\\
	$ expAux: \lambda f.\lambda n.\lambda m.\text{\underline{ift}}\ (\text{\underline{esCero?}}\ m)\ (\lambda s.\lambda z.s\ z)\ (\text{\underline{prod}}\ n\ (f\ n\ (\text{\underline{pred}}\ m)))  $\\\\
    $ \text{\textbf{FuncionExponente}}: Y\ \text{\underline{expAux}} $\\\\
    \item \textbf{Una función que decida si un natural de Church es impar.}\\\\
    $ caseN: \lambda n.\lambda a.\lambda f.(n\ a)\ f\ \\ (\text{Si 'n' es cero de Church, regresa 'a', si no regresa (f x), donde x es el predecesor de n.})$\\\\
    $ impSAux: \lambda f.\lambda n.\text{\underline{caseN}}\ n\ \text{\underline{false}}\ (\lambda n'.\text{\underline{caseN}}\ n' \ \text{\underline{true}}\ f) $\\\\
      $ \text{\textbf{FuncionImparS}}: Y\ \text{\underline{impSAux}}$\\
  \end{enumerate}
  
  \section*{5. Da una definicion inductiva mediante juicios de palabras pal\'indromas sobre el alfabeto {a, b}.}
  \begin{enumerate}[label=\alph*)]
  \item Enuncia el principio de induccion para los juicios que definiste. 
  
  $\dfrac{}{a P}$  $\dfrac{}{b P}$\\
  
$\dfrac{w P}{awa P}$  $\dfrac{w P}{bwb P}$\\

Las primeras dos son un axioma; la cadena de un s\'olo elemento es palindroma.\\
Las siguientes dos nos dice que si w es una cadena palindroma entonces al agregar una a al principio y otra al final (o una b) sigue siendo palindroma. 

  \item Demuestra por inducci\'on matem\'atica que si w es una cadena pal\'indroma entonces reverse(w) = w.
  
  \section*{7. Extiende el lenguaje EAB con un operador even que tome un natural y decida si dicho numero es par de la siguiente manera.}
$\blacksquare$Extiende la sintaxis concreta\\ \\
e ::= x	$|$ n $|$ true $|$ false $|$ e+e $|$ e*e $|$ suc e $|$ pred e $|$ if e then e else e$|$\\ iszero e $|$ iseven e $|$ let x = e in e end\\\\
$\blacksquare$Extiende la sintaxis abstracta\\\\
t ::= x 	$|$ num[n]	$|$ bool[true]	$|$ bool[false] 
		$|$ suma(t1,t2)	$|$ prod(t1,t2) 	$|$ suc(t) $|$ pred(t)
		$|$ if(t1,t2,t3) 	$|$ iszero(t) 	$|$ iseven(t)	$|$ let(t1,x.t2)\\\\
$\blacksquare$Extiende la sem\'antica est\'atica\\\\
$ \Gamma$ $\vdash$ t : Nat\\
$ \Gamma$ $\vdash$ iseven t: Bool\\\\
$\blacksquare$Extiende la sem\'antica din\'amica\\\\\\
$\dfrac{}{iseven num[0] \rightarrow bool[true]}$\\\\\\
$\dfrac{}{iseven num[1] \rightarrow bool[false]}$\\\\\\
$\dfrac{t1 \rightarrow t1'}{iseven t1 \rightarrow iseven  t1'}$\\\\\\
$\dfrac{}{iseven num[n] \rightarrow iseven[pred(pred(num[n]))]}$\\\\\\

\section*{8. Sean e1 y e2 las siguientes expresiones: \\ \\ e1 = let y = suc(2+1) in (let z = pred(5) in z+y end) * 2+y end \\ e2 = (let y = x+v in (let z = x in x*y*z end) end)[x:=y*z]}
\begin{itemize}
\item Convierte las expresiones e1 y e2 a sus respectivas representaciones como asas, t1 y t2, respectiva-
mente.

t1:=let(var[y],suc(suma(num[2],num[1])),prod(let(var[z],pred(num[5]),\\suma(var[z],var[y])),suma(num[2],var[y])))\\
t2:=let z = let(var[z],var[x],prod(var[x],prod(var[y],var[z])))\\ 
t2':=let y = let(var[y],suma(var[x],var[v]),let(var[z],var[x],prod(var[x],prod(var[y],var[z]))))
[x:=y*z]

\item Con los juicios para la semantica est\'atica haz derivaciones para t1 y t2.

e1 =let y = suc(2+1) in (let z = pred(5) in z+y end) * 2+y end \\
t1=let(var[y],suc(suma(num[2],num[1])),\\prod(let(var[z],pred(num[5]),suma(var[z],var[y])),suma(num[2],var[y])))\\\\
$\vdash$ num[2]:Nat (tnum)\\
$\vdash$ num[1]:Nat (tnum)\\
$\vdash$ suma(num[2],num[1]): Nat								(tsum)\\
$\vdash$ suc(suma(num[2],num[1])):Nat							(tsuc)\\
y:Nat $\vdash$ y:Nat										(tvar)\\
y:Nat,z:Nat $\vdash$ z:Nat									(tvar)\\
y:Nat,z:Nat $\vdash$ num[5]:Nat								(tnum)\\
y:Nat,z:Nat $\vdash$ pred(num[5]):Nat							(tpred)\\
y:Nat,z:Nat $\vdash$ suma(var[z],var[y]): Nat						(tsum)\\
y:Nat,z:Nat $\vdash$ let(var[z],pred(num[5]),suma(var[z],var[y])):Nat			(tlet)\\
y:Nat,z:Nat $\vdash$ suma(num[2],var[y]):Nat						(tsum)\\
y:Nat,z:Nat $\vdash$ prod(suma(var[z],var[y]),suma(num[2],var[y])):Nat			(tprod)\\
y:Nat,z:Nat $\vdash$let(var[y],
suc(suma(num[2],num[1])),\\
prod(let(var[z],pred(num[5]),\\suma(var[z],var[y])),suma(num[2],var[y]))):Nat	(tlet)\\\\
e2 =  (let y = x+v in (let z = x in x*y *z end) end)[x:=y *z]\\
t2 =let y = let(var[y],suma(var[x],var[v]),let(var[z],var[x],prod(var[x],prod(var[y],var[z])))) [x:=y*z]\\\\
x:T  $\vdash$ x:T\\
v:S  $\vdash$ v:S\\
x:T,v:S $\vdash$ suma(var[x],var[v]):Nat\\\\

\item Evalua t1 y t2 usando los juicios para la semantica din\'amica.

t1=let(var[y],suc(suma(num[2],num[1])),prod(let(var[z],pred(num[5]),\\suma(var[z],var[y])),suma(num[2],var[y])))\\
e1= let y = suc(2+1) in (let z = pred(5) in z+y end) * 2+y end \\\\
let y = suc(2+1) in (let z = pred(5) in z+y end) * 2+y end			(eleti)\\
let y = suc(3) in (let z = pred(5) in z+y end) * 2+y end			(eleti)\\
let y = 4 in (let z = pred(5) in z+y end) * 2+y	end				(eletf)\\
((let z = pred(5) in z+y end)*2+y)[y:=4]					=\\
(let z = pred(5) in z+4 end) * (2 + 4)						(eleti)\\
(let z = 4 in z+4) * (2+4)							(eletf)\\
(z+4)[z:=4]*(2+4)								=\\
(4+4)*(2+4)									(eprodi)(esumaf)\\
16*(2+4)									(eprodd)(esumaf)\\
16*6										(eprodf)\\
96\\\\
t2 =let y = let(var[y],suma(var[x],var[v]),let(var[z],var[x],prod(var[x],prod(var[y],var[z])))) [x:=y*z]\\
e2 =  (let y = x+v in (let z = x in x*y *z end) end)[x:=y *z]\\\\
let y = y*z+v in (let z = y*z in y*z*y*z) 					(eletf)\\
(let z = y*z in y*z*y*z) [y:=z*z+v]						=\\
let z = (z*z+v)*z in (z*z+v)*z*(z*z+v)*z					(eletf)\\
((z*z+v)*z*(z*z+v)*z*)[z:=z*z+v]\\
\end{itemize}

  \end{enumerate}

  
\end{document}
